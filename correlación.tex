% Options for packages loaded elsewhere
\PassOptionsToPackage{unicode}{hyperref}
\PassOptionsToPackage{hyphens}{url}
%
\documentclass[
]{article}
\usepackage{amsmath,amssymb}
\usepackage{iftex}
\ifPDFTeX
  \usepackage[T1]{fontenc}
  \usepackage[utf8]{inputenc}
  \usepackage{textcomp} % provide euro and other symbols
\else % if luatex or xetex
  \usepackage{unicode-math} % this also loads fontspec
  \defaultfontfeatures{Scale=MatchLowercase}
  \defaultfontfeatures[\rmfamily]{Ligatures=TeX,Scale=1}
\fi
\usepackage{lmodern}
\ifPDFTeX\else
  % xetex/luatex font selection
\fi
% Use upquote if available, for straight quotes in verbatim environments
\IfFileExists{upquote.sty}{\usepackage{upquote}}{}
\IfFileExists{microtype.sty}{% use microtype if available
  \usepackage[]{microtype}
  \UseMicrotypeSet[protrusion]{basicmath} % disable protrusion for tt fonts
}{}
\makeatletter
\@ifundefined{KOMAClassName}{% if non-KOMA class
  \IfFileExists{parskip.sty}{%
    \usepackage{parskip}
  }{% else
    \setlength{\parindent}{0pt}
    \setlength{\parskip}{6pt plus 2pt minus 1pt}}
}{% if KOMA class
  \KOMAoptions{parskip=half}}
\makeatother
\usepackage{xcolor}
\usepackage[margin=1in]{geometry}
\usepackage{color}
\usepackage{fancyvrb}
\newcommand{\VerbBar}{|}
\newcommand{\VERB}{\Verb[commandchars=\\\{\}]}
\DefineVerbatimEnvironment{Highlighting}{Verbatim}{commandchars=\\\{\}}
% Add ',fontsize=\small' for more characters per line
\usepackage{framed}
\definecolor{shadecolor}{RGB}{248,248,248}
\newenvironment{Shaded}{\begin{snugshade}}{\end{snugshade}}
\newcommand{\AlertTok}[1]{\textcolor[rgb]{0.94,0.16,0.16}{#1}}
\newcommand{\AnnotationTok}[1]{\textcolor[rgb]{0.56,0.35,0.01}{\textbf{\textit{#1}}}}
\newcommand{\AttributeTok}[1]{\textcolor[rgb]{0.13,0.29,0.53}{#1}}
\newcommand{\BaseNTok}[1]{\textcolor[rgb]{0.00,0.00,0.81}{#1}}
\newcommand{\BuiltInTok}[1]{#1}
\newcommand{\CharTok}[1]{\textcolor[rgb]{0.31,0.60,0.02}{#1}}
\newcommand{\CommentTok}[1]{\textcolor[rgb]{0.56,0.35,0.01}{\textit{#1}}}
\newcommand{\CommentVarTok}[1]{\textcolor[rgb]{0.56,0.35,0.01}{\textbf{\textit{#1}}}}
\newcommand{\ConstantTok}[1]{\textcolor[rgb]{0.56,0.35,0.01}{#1}}
\newcommand{\ControlFlowTok}[1]{\textcolor[rgb]{0.13,0.29,0.53}{\textbf{#1}}}
\newcommand{\DataTypeTok}[1]{\textcolor[rgb]{0.13,0.29,0.53}{#1}}
\newcommand{\DecValTok}[1]{\textcolor[rgb]{0.00,0.00,0.81}{#1}}
\newcommand{\DocumentationTok}[1]{\textcolor[rgb]{0.56,0.35,0.01}{\textbf{\textit{#1}}}}
\newcommand{\ErrorTok}[1]{\textcolor[rgb]{0.64,0.00,0.00}{\textbf{#1}}}
\newcommand{\ExtensionTok}[1]{#1}
\newcommand{\FloatTok}[1]{\textcolor[rgb]{0.00,0.00,0.81}{#1}}
\newcommand{\FunctionTok}[1]{\textcolor[rgb]{0.13,0.29,0.53}{\textbf{#1}}}
\newcommand{\ImportTok}[1]{#1}
\newcommand{\InformationTok}[1]{\textcolor[rgb]{0.56,0.35,0.01}{\textbf{\textit{#1}}}}
\newcommand{\KeywordTok}[1]{\textcolor[rgb]{0.13,0.29,0.53}{\textbf{#1}}}
\newcommand{\NormalTok}[1]{#1}
\newcommand{\OperatorTok}[1]{\textcolor[rgb]{0.81,0.36,0.00}{\textbf{#1}}}
\newcommand{\OtherTok}[1]{\textcolor[rgb]{0.56,0.35,0.01}{#1}}
\newcommand{\PreprocessorTok}[1]{\textcolor[rgb]{0.56,0.35,0.01}{\textit{#1}}}
\newcommand{\RegionMarkerTok}[1]{#1}
\newcommand{\SpecialCharTok}[1]{\textcolor[rgb]{0.81,0.36,0.00}{\textbf{#1}}}
\newcommand{\SpecialStringTok}[1]{\textcolor[rgb]{0.31,0.60,0.02}{#1}}
\newcommand{\StringTok}[1]{\textcolor[rgb]{0.31,0.60,0.02}{#1}}
\newcommand{\VariableTok}[1]{\textcolor[rgb]{0.00,0.00,0.00}{#1}}
\newcommand{\VerbatimStringTok}[1]{\textcolor[rgb]{0.31,0.60,0.02}{#1}}
\newcommand{\WarningTok}[1]{\textcolor[rgb]{0.56,0.35,0.01}{\textbf{\textit{#1}}}}
\usepackage{longtable,booktabs,array}
\usepackage{calc} % for calculating minipage widths
% Correct order of tables after \paragraph or \subparagraph
\usepackage{etoolbox}
\makeatletter
\patchcmd\longtable{\par}{\if@noskipsec\mbox{}\fi\par}{}{}
\makeatother
% Allow footnotes in longtable head/foot
\IfFileExists{footnotehyper.sty}{\usepackage{footnotehyper}}{\usepackage{footnote}}
\makesavenoteenv{longtable}
\usepackage{graphicx}
\makeatletter
\def\maxwidth{\ifdim\Gin@nat@width>\linewidth\linewidth\else\Gin@nat@width\fi}
\def\maxheight{\ifdim\Gin@nat@height>\textheight\textheight\else\Gin@nat@height\fi}
\makeatother
% Scale images if necessary, so that they will not overflow the page
% margins by default, and it is still possible to overwrite the defaults
% using explicit options in \includegraphics[width, height, ...]{}
\setkeys{Gin}{width=\maxwidth,height=\maxheight,keepaspectratio}
% Set default figure placement to htbp
\makeatletter
\def\fps@figure{htbp}
\makeatother
\setlength{\emergencystretch}{3em} % prevent overfull lines
\providecommand{\tightlist}{%
  \setlength{\itemsep}{0pt}\setlength{\parskip}{0pt}}
\setcounter{secnumdepth}{-\maxdimen} % remove section numbering
\ifLuaTeX
  \usepackage{selnolig}  % disable illegal ligatures
\fi
\IfFileExists{bookmark.sty}{\usepackage{bookmark}}{\usepackage{hyperref}}
\IfFileExists{xurl.sty}{\usepackage{xurl}}{} % add URL line breaks if available
\urlstyle{same}
\hypersetup{
  pdftitle={correlación},
  pdfauthor={Manuel Martín},
  hidelinks,
  pdfcreator={LaTeX via pandoc}}

\title{correlación}
\author{Manuel Martín}
\date{2024-02-26}

\begin{document}
\maketitle

\#\#\#Ejercicio 1

\begin{Shaded}
\begin{Highlighting}[]
\FunctionTok{library}\NormalTok{(readxl)}
\NormalTok{data }\OtherTok{\textless{}{-}} \FunctionTok{as.data.frame}\NormalTok{(}\FunctionTok{read\_excel}\NormalTok{(}\StringTok{"C:/Users/manueltristan/Documents/correlacion/data.xlsx"}\NormalTok{))}
\FunctionTok{View}\NormalTok{(data)}
\FunctionTok{print}\NormalTok{(data)}
\end{Highlighting}
\end{Shaded}

\begin{verbatim}
##    longitud ancho grosor   peso
## 1      12.4   3.6  17.36  167.0
## 2      22.6   4.3  21.82  342.1
## 3      17.9   4.1  13.54  322.9
## 4      10.2  10.2  40.90  154.8
## 5      16.8   5.7  34.06  358.1
## 6      13.3   4.1  35.36  227.9
## 7      14.1   5.8 108.64  323.8
## 8      10.2   5.9 125.64  285.2
## 9      22.5   6.2  80.20  613.8
## 10     16.9   3.6  60.48  254.3
## 11     19.1   4.1 124.70  310.1
## 12     25.8   4.7 195.78  426.8
## 13     22.5   3.9 121.58  521.2
## 14     27.6  10.2  33.12  765.1
## 15     38.0  10.2  61.58 1217.2
## 16     72.4   6.4  38.48 2446.5
## 17     37.5   3.9 104.94  675.7
## 18     10.2   2.7  22.24   90.9
## 19     11.6   2.0  35.74   86.8
## 20     10.8   2.7  54.68  109.1
## 21     11.4   1.8 260.88   67.7
## 22     10.2   2.8  46.76  204.5
## 23     10.2   3.3   0.00  170.3
## 24     18.6   2.7   0.00  176.8
## 25     24.4   4.4   0.00  543.2
## 26     23.5   4.5   0.00  628.2
## 27     24.8   3.5   0.00  401.0
## 28     14.1   3.9   0.00  302.4
## 29     24.6   4.8   0.00  623.5
## 30     30.9   6.0   0.00  978.9
## 31     20.2   5.7   0.00  607.9
## 32     12.8   2.8   0.00  165.6
## 33     16.9   3.6   0.00  307.9
## 34     14.2   2.8   0.00  192.4
## 35     18.0   5.3   0.00  524.7
## 36     11.7   2.4   0.00  111.2
## 37     14.1   2.4   0.00  178.7
## 38     17.7   3.9   0.00  273.4
## 39     36.6   6.0   0.00 1304.4
## 40     12.3   5.4   0.00  233.8
\end{verbatim}

\begin{Shaded}
\begin{Highlighting}[]
\CommentTok{\#Función para agregar coeficientes de correlación}
\NormalTok{panel.cor }\OtherTok{\textless{}{-}} \ControlFlowTok{function}\NormalTok{(x, y, }\AttributeTok{digits =} \DecValTok{2}\NormalTok{, }\AttributeTok{prefix =} \StringTok{""}\NormalTok{, cex.cor, ...) \{}
\NormalTok{  usr }\OtherTok{\textless{}{-}} \FunctionTok{par}\NormalTok{(}\StringTok{"usr"}\NormalTok{)}
  \FunctionTok{on.exit}\NormalTok{(}\FunctionTok{par}\NormalTok{(usr))}
  \FunctionTok{par}\NormalTok{(}\AttributeTok{usr =} \FunctionTok{c}\NormalTok{(}\DecValTok{0}\NormalTok{, }\DecValTok{1}\NormalTok{, }\DecValTok{0}\NormalTok{ ,}\DecValTok{1}\NormalTok{))}
\NormalTok{  Cor }\OtherTok{\textless{}{-}} \FunctionTok{abs}\NormalTok{(}\FunctionTok{cor}\NormalTok{(x, y)) }
\NormalTok{  txt }\OtherTok{\textless{}{-}} \FunctionTok{paste0}\NormalTok{(prefix, }\FunctionTok{format}\NormalTok{(}\FunctionTok{c}\NormalTok{(Cor, }\FloatTok{0.123456789}\NormalTok{), }\AttributeTok{digits =}\NormalTok{ digits)[}\DecValTok{1}\NormalTok{])}
  \ControlFlowTok{if}\NormalTok{(}\FunctionTok{missing}\NormalTok{(cex.cor)) \{}
\NormalTok{    cex.cor }\OtherTok{\textless{}{-}} \FloatTok{0.4} \SpecialCharTok{/} \FunctionTok{strwidth}\NormalTok{(txt)}
\NormalTok{  \}}
  \FunctionTok{text}\NormalTok{(}\FloatTok{0.5}\NormalTok{, }\FloatTok{0.5}\NormalTok{, txt,}
       \AttributeTok{cex =} \DecValTok{1} \SpecialCharTok{+}\NormalTok{ cex.cor}\SpecialCharTok{*}\NormalTok{Cor)}
\NormalTok{\}}
\CommentTok{\#Dibujamos la matriz de correlación}
\FunctionTok{pairs}\NormalTok{(data,}
      \AttributeTok{upper.panel =}\NormalTok{ panel.cor, }\CommentTok{\# Panel de correlación}
      \AttributeTok{lower.panel =}\NormalTok{ panel.smooth) }\CommentTok{\#Curvas de regresión suavizadas}
\end{Highlighting}
\end{Shaded}

\begin{verbatim}
## Warning in par(usr): argument 1 does not name a graphical parameter

## Warning in par(usr): argument 1 does not name a graphical parameter

## Warning in par(usr): argument 1 does not name a graphical parameter

## Warning in par(usr): argument 1 does not name a graphical parameter

## Warning in par(usr): argument 1 does not name a graphical parameter

## Warning in par(usr): argument 1 does not name a graphical parameter
\end{verbatim}

\includegraphics{correlación_files/figure-latex/unnamed-chunk-2-1.pdf}

\begin{Shaded}
\begin{Highlighting}[]
\FunctionTok{cor.test}\NormalTok{(data}\SpecialCharTok{$}\NormalTok{longitud, data}\SpecialCharTok{$}\NormalTok{peso)}
\end{Highlighting}
\end{Shaded}

\begin{verbatim}
## 
##  Pearson's product-moment correlation
## 
## data:  data$longitud and data$peso
## t = 19.989, df = 38, p-value < 2.2e-16
## alternative hypothesis: true correlation is not equal to 0
## 95 percent confidence interval:
##  0.9170685 0.9764377
## sample estimates:
##       cor 
## 0.9555894
\end{verbatim}

\begin{Shaded}
\begin{Highlighting}[]
\FunctionTok{library}\NormalTok{(correlation)}
\NormalTok{resultados }\OtherTok{\textless{}{-}} \FunctionTok{correlation}\NormalTok{(data)}
\NormalTok{resultados}
\end{Highlighting}
\end{Shaded}

\begin{verbatim}
## # Correlation Matrix (pearson-method)
## 
## Parameter1 | Parameter2 |         r |        95% CI |     t(38) |         p
## ---------------------------------------------------------------------------
## longitud   |      ancho |      0.40 | [ 0.10, 0.63] |      2.71 | 0.040*   
## longitud   |     grosor |  4.68e-03 | [-0.31, 0.32] |      0.03 | > .999   
## longitud   |       peso |      0.96 | [ 0.92, 0.98] |     19.99 | < .001***
## ancho      |     grosor | -1.29e-03 | [-0.31, 0.31] | -7.98e-03 | > .999   
## ancho      |       peso |      0.51 | [ 0.23, 0.71] |      3.64 | 0.004**  
## grosor     |       peso |     -0.06 | [-0.36, 0.26] |     -0.36 | > .999   
## 
## p-value adjustment method: Holm (1979)
## Observations: 40
\end{verbatim}

\begin{Shaded}
\begin{Highlighting}[]
\FunctionTok{library}\NormalTok{(ggpubr)}
\end{Highlighting}
\end{Shaded}

\begin{verbatim}
## Loading required package: ggplot2
\end{verbatim}

\begin{Shaded}
\begin{Highlighting}[]
\FunctionTok{library}\NormalTok{(ggplot2)}
\FunctionTok{ggscatter}\NormalTok{(data, }\AttributeTok{x =} \StringTok{"longitud"}\NormalTok{, }\AttributeTok{y =} \StringTok{"peso"}\NormalTok{,}
          \AttributeTok{add =} \StringTok{"reg.line"}\NormalTok{, }\AttributeTok{conf.int =} \ConstantTok{TRUE}\NormalTok{,}
          \AttributeTok{cor.coef =} \ConstantTok{TRUE}\NormalTok{, }\AttributeTok{cor.method =} \StringTok{"pearson"}\NormalTok{,}
          \AttributeTok{xlab =} \StringTok{"longitud piezas (mm)"}\NormalTok{, }\AttributeTok{ylab =} \StringTok{"Peso piezas (mg)"}\NormalTok{)}
\end{Highlighting}
\end{Shaded}

\includegraphics{correlación_files/figure-latex/unnamed-chunk-5-1.pdf}

\begin{Shaded}
\begin{Highlighting}[]
\FunctionTok{library}\NormalTok{(corrplot)}
\end{Highlighting}
\end{Shaded}

\begin{verbatim}
## corrplot 0.92 loaded
\end{verbatim}

\begin{Shaded}
\begin{Highlighting}[]
\FunctionTok{corrplot}\NormalTok{(}\FunctionTok{cor}\NormalTok{(data))}
\end{Highlighting}
\end{Shaded}

\includegraphics{correlación_files/figure-latex/unnamed-chunk-6-1.pdf}

\begin{Shaded}
\begin{Highlighting}[]
\FunctionTok{library}\NormalTok{(corrplot)}
\FunctionTok{corrplot.mixed}\NormalTok{(}\FunctionTok{cor}\NormalTok{(data))}
\end{Highlighting}
\end{Shaded}

\includegraphics{correlación_files/figure-latex/unnamed-chunk-7-1.pdf}
\#\#\# Ejercicio VI

A partir de las siguientes variables

\begin{itemize}
\tightlist
\item
  Distancia (km): 1.1,100.2,90.3,5.4,57.5,6.6,34.7,65.8,57.9,86.1
\item
  Número de cuentas (valor absoluto): 110,2,6,98,40,94,31,5,8,10
\end{itemize}

\begin{enumerate}
\def\labelenumi{\alph{enumi}.}
\tightlist
\item
  Crea 2 vectores: ``distancia'' y ``n\_piezas''
\end{enumerate}

\begin{Shaded}
\begin{Highlighting}[]
\NormalTok{distancia }\OtherTok{\textless{}{-}} \FunctionTok{c}\NormalTok{(}\FloatTok{1.1}\NormalTok{,}\FloatTok{100.2}\NormalTok{,}\FloatTok{90.3}\NormalTok{,}\FloatTok{5.4}\NormalTok{,}\FloatTok{57.5}\NormalTok{,}\FloatTok{6.6}\NormalTok{,}\FloatTok{34.7}\NormalTok{,}\FloatTok{65.8}\NormalTok{,}\FloatTok{57.9}\NormalTok{,}\FloatTok{86.1}\NormalTok{)}
\NormalTok{n\_piezas }\OtherTok{\textless{}{-}} \FunctionTok{c}\NormalTok{(}\DecValTok{110}\NormalTok{,}\DecValTok{2}\NormalTok{,}\DecValTok{6}\NormalTok{,}\DecValTok{98}\NormalTok{,}\DecValTok{40}\NormalTok{,}\DecValTok{94}\NormalTok{,}\DecValTok{31}\NormalTok{,}\DecValTok{5}\NormalTok{,}\DecValTok{8}\NormalTok{,}\DecValTok{10}\NormalTok{)}
\NormalTok{dist\_ncuent }\OtherTok{\textless{}{-}} \FunctionTok{data.frame}\NormalTok{(distancia,n\_piezas)}
\NormalTok{knitr}\SpecialCharTok{::}\FunctionTok{kable}\NormalTok{(dist\_ncuent)}
\end{Highlighting}
\end{Shaded}

\begin{longtable}[]{@{}rr@{}}
\toprule\noalign{}
distancia & n\_piezas \\
\midrule\noalign{}
\endhead
\bottomrule\noalign{}
\endlastfoot
1.1 & 110 \\
100.2 & 2 \\
90.3 & 6 \\
5.4 & 98 \\
57.5 & 40 \\
6.6 & 94 \\
34.7 & 31 \\
65.8 & 5 \\
57.9 & 8 \\
86.1 & 10 \\
\end{longtable}

\begin{enumerate}
\def\labelenumi{\alph{enumi}.}
\setcounter{enumi}{1}
\tightlist
\item
  Calcula el coeficiente de correlación La correlación es de -0.9249824
\item
  calcula el nivel de significacncia El nivel de significancia es
  0.0001265
\item
  Intervalo de confianza al 95\% en relación con el coeficiente de
  correlación. El intervalo de confianza al 95\% es {[}-0.9824414,
  -0.7072588{]}.
\item
  ¿Que intensidad y dirección presentan ambas variables? Al ser un
  coeficiente negativo, la dirección es inversa entre ambas variables y
  su intensidad es de 0.9249824, bastante cercana al máximo que es 1
\item
  ¿Es significativa esta relación? Si la relación es significativa pues
  el nivel de significancia es inferior a 0.05 o lo que es lo mismo
  inferior al 5\% de posibilidades de que esta relación sea resultado
  del azar.
\item
  ¿Resultaría apropiado afirmar la correlación (o no) entre variables
  con un tamaño muestral tan reducido (n=10)? Los datos muestran que es
  una correlación significativa, por lo que cabría experar que aunque
  aumenten el número de muestras, no debería haber una variación
  significativa. Porque hemos comprobado que la correlación no se debe
  al azar sino que responde a una correlación lineal, que se debería
  mantener conforme aumente el número de muestras.
\end{enumerate}

\begin{Shaded}
\begin{Highlighting}[]
\FunctionTok{cor.test}\NormalTok{(dist\_ncuent}\SpecialCharTok{$}\NormalTok{distancia, dist\_ncuent}\SpecialCharTok{$}\NormalTok{n\_piezas)}
\end{Highlighting}
\end{Shaded}

\begin{verbatim}
## 
##  Pearson's product-moment correlation
## 
## data:  dist_ncuent$distancia and dist_ncuent$n_piezas
## t = -6.8847, df = 8, p-value = 0.0001265
## alternative hypothesis: true correlation is not equal to 0
## 95 percent confidence interval:
##  -0.9824414 -0.7072588
## sample estimates:
##        cor 
## -0.9249824
\end{verbatim}

\begin{Shaded}
\begin{Highlighting}[]
\NormalTok{correlation}\SpecialCharTok{::}\FunctionTok{correlation}\NormalTok{(dist\_ncuent)}
\end{Highlighting}
\end{Shaded}

\begin{verbatim}
## # Correlation Matrix (pearson-method)
## 
## Parameter1 | Parameter2 |     r |         95% CI |  t(8) |         p
## --------------------------------------------------------------------
## distancia  |   n_piezas | -0.92 | [-0.98, -0.71] | -6.88 | < .001***
## 
## p-value adjustment method: Holm (1979)
## Observations: 10
\end{verbatim}

\end{document}
